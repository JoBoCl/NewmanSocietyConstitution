\documentclass[11pt]{article}

\usepackage{mathpazo}

%\usepackage{quattrocento}
\usepackage[T1]{fontenc}

\usepackage[normalem]{ulem}
\usepackage[margin=1.25in]{geometry}
\usepackage[colorlinks,linkcolor=blue]{hyperref}
\usepackage{titlesec}
\usepackage{titletoc}
\usepackage{paralist}
\usepackage[nameinlink]{cleveref}
\usepackage{draftwatermark}
\usepackage{ifthen}


\crefname{enumii}{subitem}{subitems}
\renewcommand{\labelenumii}{(\roman{enumii})}

%\creflabelformat{enumii}{#2(#1)#3}

\date{}
\usepackage[protrusion,expansion,final,tracking=true,kerning=true,spacing=true,factor=1100,stretch=10,shrink=10]{microtype}

\title{\scshape{The Constitution of the Newman Society: Oxford University Catholic Society}}

%\titlelabel{\thetitle\quad}
\titleformat{\section}{\normalfont\large\scshape}{}{0.5em}{\hrule height 1pt\vspace{2pt}
%supress section number in contents
\ifthenelse{\boolean{contents}}{\thesection --}{}}[\hrule height 0.2pt]
\titlespacing*{\section}{0pt}{0pt}{10pt}

\newboolean{contents}


\titleformat{\subsection}{\normalfont\scshape}{}{0.5em}{\hrule height 0.2pt\vspace{1pt}\thesubsection --}[\hrule height 0.2pt]
\titlespacing*{\section}
  {0pt}{0pt}{5pt}


\begin{document}
\renewcommand{\theenumi}{\Alph{enumi}}
\renewcommand{\theenumii}{(\roman{enumii})}
\setcounter{tocdepth}{1}
\maketitle
\centerline{\scshape{AMDG -- Ad Maiorem Dei Gloriam}}

\centerline{\scshape{For the greater glory of God}}
\setboolean{contents}{false}
\tableofcontents
\setboolean{contents}{true}
\section{Name and objects}\label{sec:nam}
\begin{enumerate}
\item \label{nam:aim}The Society is called the Newman Society: Oxford University Catholic Society. The Society's objects are the support, development, improvement and promotion of to work in conjunction with the Chaplains to support and encourage Catholic students in their Christian vocation by promoting their personal, intellectual and spiritual development, social interaction, and apostolic witness within the broad context of their university experience.  It also seeks to establish and maintain links with other Catholic organisations within the University of Oxford or outside of the University of Oxford but within Oxford. It also seeks to establish and maintain links with other religious groups within the University. The income and property of the Society shall be applied solely to those objects.  The address of the Society shall be: \\
The Newman Society: Oxford University Catholic Society\\
The Old Palace\\
Rose Place\\
St Aldate's\\
OXFORD\\
OX1 1RD

The history of the society is as follows:
\begin{enumerate}
\item \label{his:fou} The Newman Society was founded in 1878 as a society for the Catholic Student members of the University of Oxford.  Its name was adopted in honour of Blessed John Henry Cardinal Newman with his consent. The name provides a connexion of living tradition between Catholic students of the University and Blessed John Henry and is one of the treasures of Catholic Oxford. In 2009, the St Thomas More Lectures were founded by the Newman Society to bring persons of international eminence to lecture in Oxford on subjects of Catholic interest. The inaugural lecture was given by His Eminence George Cardinal Pell, Archbishop of Sydney.
\item \label{his:rep} At the beginning of Trinity Term 1990, the Roman Catholic Chaplains to the University of Oxford invited those students currently serving as Chaplaincy representatives in their colleges and those students representing the various groups attached to the Chaplaincy to discuss with them the setting up a further means of promoting the well-being and apostolate of the Catholic student body throughout the University. As a result, the University Catholic Society was brought into being.
\item \label{his:con} At a Meeting of the Committee held at the Old Palace, the following resolution was passed: With provision for future amendments as deemed necessary, this Constitution is formally approved by the Committee as binding upon the Society and its members, and is deposited with the Chaplains in Hilary Term 1991.
\item On April 28th, 1993, the Vice-Chancellor's approval was secured for the Society to be known as the Oxford University Catholic Society. At a Meeting of the Full Committee held at the Old Palace on 15 October 1995 the Constitution was amended to take account of this and other changes from experience gained in the first five years of the Catholic Society's existence.
\item On November 11th, 1999, the University Catholic Chaplaincy Social Club became a legal entity in whose name the bar licence was held. At a meeting of the Full Committee held at the Catholic Chaplaincy on January 16, 2000, the University Catholic Chaplaincy Social Club was included within the Constitution of the Oxford University Catholic Society, and relevant changes were made to the Oxford University Catholic Society's Constitution. On November 9th , 2012, as part of the revisions to this Constitution following the vote of Union with the Newman Society (cf. \cref{his:join}) a vote was taken to end the existence of the defunct University Catholic Chaplaincy Social Club. 
\item \label{his:join} In 2012, the Newman Society and the Oxford University Catholic Society agreed (cf., \cref{his:fou}) to join together as a single society for Catholic members of the University giving witness in the University to the unity and catholicity of the Church (cf., \cref{his:rep}) to preserve the name of the Newman Society and (cf., \cref{his:con}) to make provision for the continuation of the St Thomas More Lectures by merging to create one Society to be known as ``The Newman Society: Oxford University Catholic Society''.
\end{enumerate}
\end{enumerate}
\section{Compliance}\label{con:com}
\begin{enumerate}
\item The Society shall be administered in accordance with the regulations for University clubs which are published from time to time in the Proctors' and Assessor's Memorandum (``the Proctors' Memorandum'').
\item The activities of the Society will at all times be conducted in accordance with the following University policies and codes of practice in force from time to time: Integrated Equality Policy, Code of Practice on Harassment and Bullying, and Code of Practice on Freedom of Speech.
\item \label{com:ins} If there is a national governing body for the Society's activities with which the Society is eligible to register, the Society shall effect and maintain such registration: purchase any insurance cover which the national body makes available unless the Insurance Section of the University's Central Administration (``the Insurance Section'') agrees to or prescribes other arrangements; and make every effort to comply with all safety procedures which the national body prescribes, or recommends as good practice.
\subitem As of the drafting of this Constitution, the Society is aware of no such relevant national body.
\item The Society shall observe the Code of Conduct on Safety Matters which is set out in the Schedule to this Constitution, ensure compliance with the Code by the members of the Society, and follow an appropriate procedure for risk assessment. Both the Code of Conduct and the procedure for risk assessment must be acceptable to the University's Safety Officer.
\item No member of the Society shall participate in any activity overseas organised by the Society, whether during term-time or vacation, unless the plans for such activity have been notified at least one calendar month in advance of the date of departure from the United Kingdom to the University Marshal. Each member participating in such activities overseas shall observe any conditions imposed by the Proctors on the recommendation of the University Marshal, e.g. relating to the deposit of contact addresses, fulfilment of health, safety and insurance requirements, and stipulation of Senior Members to accompany the trip.
\item  The Society may apply to IT Services, University of Oxford to use information technology (`IT') facilities in the name of the Society. Where relevant facilities are allocated by IT Services it is the responsibility of the Society:
\begin{enumerate}
\item \label{it:des} to designate a member of the Society entitled to a University e-mail account (as defined by IT Services rules) to act as its IT Officer, whose duties shall include liaising with IT Services about the use of facilities allocated and passing on to his or her successor in office all records relating to the use of the facilities allocated;
\item to designate one of its members (who may be, but need not necessarily be, the same as its IT Officer) or, exceptionally, a member of Congregation to act as its principal Webmaster, whose duties shall include maintaining an awareness of the University Guidelines for Web Information Providers and co-ordinating and regulating access to the web facilities use by the Society;
\item to comply with regulations and guidelines relating to the use of IT facilities published from time to time by IT Services;
\item \label{it:res}to ensure that everyone responsible under \crefrange{it:des}{it:res} is competent to deal with the requirements, where necessary undertaking training under the guidance of IT Services.
\end{enumerate}
\end{enumerate}
\section{Membership}\label{con:mem}
\begin{enumerate}
\item The members of the Society shall be those who are eligible and apply for membership of the Society, who are admitted to and maintained in membership by the Committee.  There are not to be any fees dues, or other monetary subscriptions required for membership in the society.
\item \label{mem:bap} All student members of the University who have been baptised into the Roman Catholic Church shall be eligible to become members of the Society. A member shall continue to be eligible until he or she is given permission to supplicate for his or her degree, diploma, or certificate, regardless of whether or not he or she may continue to be liable to pay fees to the University.
\item The Committee may also, at its discretion, admit to membership any baptised Roman Catholics of the following distinctions:
\begin{enumerate}
\item \label{mem:vis} any whose names are on the University's Register of Visiting Students;
\item any students registered to read for diplomas and certificates in the University;
\item \label{mem:non} any student members of Permanent Private Halls who are not student members of the University;
\item \label{mem:oth} and any other person not falling within sections~\cref{mem:bap} and \crefrange{mem:vis}{mem:non}, above -- including those in the catechumenate or otherwise not members of the Roman Catholic Church -- provided that such members shall not constitute more than one-fifth of the total membership of the Society.
\end{enumerate}
\item The Committee may remove a person from membership for good cause. The person concerned may appeal against such removal to the chaplains.  The expelled member shall have recourse to the Senior Member if desigring to appeal against such a removal of membership.
\end{enumerate}
\section{Meetings of the Members}
\begin{enumerate}
\item There shall be an Annual General Meeting for all the members of the Society in Michaelmas Full Term, convened by the Secretary on not less than twenty-one days' notice.
\item The Annual General Meeting will:
\begin{enumerate}
\item receive the annual report of the Committee for the previous year and the annual accounts of the Society for the previous year, the report and accounts having been approved by the Committee;
\item receive a report from the Committee on the Society's compliance with paragraph~2 above;
\item elect Members of the Committee in accordance with paragraph~19 below;
\item consider any motions of which due notice has been given, and any other relevant business.
\end{enumerate}
\item An Extraordinary General Meeting may be called in any Full Term; by the committee on not less than seven days' notice; or on a written requisition by seven or more members, stating the reason for which the meeting is to be called, and delivered to the Secretary not less than twenty-one days before the date of the Meeting.
\item Prior to all General Meetings notice of the agenda shall be sent out with the notice of the Meeting.
\item The quorum for a General Meeting shall be twice the number of members of the committee plus one present in person. 
\end{enumerate}
\section{The Committee}\label{con:str}\label{con:mee}
\begin{enumerate}
\item The affairs of the Society shall be administered by a Committee consisting of not more than nine full members of the society, and have ultimate responsibility for the activities of the Society. Members of the University shall at all times make up the majority of the members of the Committee. The Committee shall have control of the funds and property of the Society, and of its administration.
\item The quorum for a Committee meeting shall be seven members present in person. 
\item The Committee shall be made up of the President, the Secretary, the Treasurer (together, the ``Office Holders''; and their offices are referred to as ``the Offices''), the Senior Member and six other persons. The President, the Secretary and the Treasurer shall each be either a member of the Society whose eligibility stems from \nameref{con:mem} (\cref{con:mem} \cref{mem:bap}) above or \nameref{con:mem} (\cref{con:mem} \crefrange{mem:vis}{mem:oth}) above, or (with the approval of the Proctors) a member of Congregation. If his or her eligibility stems from \nameref{con:mem} (\cref{con:mem} \crefrange{mem:vis}{mem:oth}) above, on election to office he or she must sign an undertaking to abide by the Proctors' Memorandum, and to accept the authority of the Proctors on Society matters.
\item The President shall have the right to preside at all meetings of the members of the Society and at all meetings of the Committee. Should the President be absent, or decline to take the chair, the Committee shall elect another member of the Committee to chair the meeting.
\item \label{com:sec} The Secretary shall:
\begin{enumerate}
\item maintain a register of the members of the Society, which shall be available for inspection by the Proctors on request;
\item give notice of meetings of the members and the Committee;
\item draw up the agendae for and the minutes of those meetings;
\item notify the Proctors promptly following the appointment and resignation or removal of Office Holders and other members of the Committee;
\item advise the Proctors promptly of any changes in this Constitution;
\item notify the Proctors not later than the end of the second week of every Full Term of the programme of meetings which has been arranged for that term (e.g. by providing them a copy of the term card); 
\item provide the Insurance Section with full details of any insurance cover purchased from or through a national governing body pursuant to \nameref{con:com} (\cref{con:com} \cref{com:ins}) above; and
\item inform the Proctors if the Society ceases to operate, or is to be dissolved, and in doing so present a final statement of accounts (the format of which the Proctors may prescribe).
\end{enumerate}
\item \label{com:tre} The Treasurer shall:
\begin{enumerate}
\item \label{tre:rec} keep proper records of the Society's financial transactions in accordance with current accepted accounting rules and practices;
\item \label{tre:aud} develop and implement control procedures to minimise the risk of financial exposure, such procedures to be reviewed regularly with the University's Internal Audit Section (``Internal Audit'');
\item ensure that bills are paid and cash is banked in accordance with the procedures developed under \cref{tre:aud};
\item prepare an annual budget for the Society, and regularly inform the Committee of progress against that budget;
\item ensure that all statutory returns are made including VAT, income tax and corporation tax if appropriate;
\item seek advice as necessary on tax matters from the University's Finance Division;
\item develop and maintain a manual of written procedures for all aspects of the Treasurer's responsibilities;
\item make all records, procedures and accounts available on request to the Senior Member, the Proctors and Internal Audit;
\item forward to the Proctors by the end of the second week of each Full Term a copy of the accounts for the preceding term (the format of which the Proctors may prescribe) signed by the Senior Member, for retention on the Proctors' files; and
\item if the Society has a turnover in excess of \pounds 15,000 in the preceding year, or if owing to a change in the nature or scale of its activities, it may confidently be expected to have such a turnover in the current year, submit its accounts (the format of which the Proctors may prescribe) for independent professional inspection and report by a reporting accountant approved in advance by the Proctors. Accounts are to be ready for inspection within four months of the end of the Society's financial year and the costs of the inspection and report shall be borne by the Society. If requested by the reporting accountant, the Society shall submit accounts and related material as a basis for a review of accounting procedures, the cost likewise to be borne by the Society.
\end{enumerate}
\item The Senior Member shall:-
\begin{enumerate}
\item hear appeals from removal from membership under paragraph~6 above;
\item following \cref{tre:rec} above, consider the accounts of the Society and sign them if he or she considers them to be in order;
\item ensure that adequate advice and assistance is available to the Secretary and the Treasurer in the performance of their responsibilities under \crefrange{com:sec}{com:tre} above; and
\item be available to represent and speak for the Society in the public forum, and before the Courts of the University and the University authorities.
\end{enumerate}
\item The members of the Committee shall be elected by the members of the Society annually, and shall be eligible for re-election. The members of the Society shall not appoint several individuals jointly to hold any of the Offices, nor allow any individual to hold more than one Office at a time. When electing other members of the Committee each year, the members of the Society shall also appoint a member of Congregation as the Senior Member, and he or she will then be a member of the Committee \emph{ex officio}.
\item If during the period between the annual elections to offices any vacancies occur amongst the members of Committee, the Committee shall have the power of filling the vacancy or vacancies up to the next Annual General Meeting by co-optation.
\item Each Office Holder must, on relinquishing his or her appointment, promptly hand to his or her successor in Office (or to another member of the Society nominated by the Committee) all official documents and records belonging to the Society, together with (on request from the Committee) any other property of the Society which may be in his or her possession; and must complete any requirements to transfer authority relating to control of the Society's bank accounts, building society accounts, or other financial affairs.
\item Without derogating from its primary responsibility, the Committee may delegate its functions to finance and general purposes and other subcommittees which are made up exclusively of members of the Committee.
\item The Committee shall have power to make regulations and by-laws in order to implement the paragraphs of this Constitution, and to settle any disputed points not otherwise provided for in this Constitution. Any alteration to this Constitution shall require the approving vote of two-thirds of those present in person at a General Meeting.
\item No member of the Committee shall be removed from office except by a majority vote of the Committee, and the unanimous consent of the Chaplains, and the Senior Member. The removed member, within two days following his removal, may appeal for a meeting of the general Society by instructing the Senior Member to call a general Society meeting. The removed member will be restored to their former position by the approving votes of two-thirds of those present in person at a general meeting.
\end{enumerate}
\section{Indemnity}
\begin{enumerate}
\item So far as may be permitted by law, every member of the Committee and every officer of the Society shall be entitled to be indemnified by the Society against all costs, charges, losses, expenses and liabilities incurred by him or her in the execution or discharge of his or her duties or the exercise of his or her powers, or otherwise properly in relation to or in connection with his or her duties. This indemnity extends to any liability incurred by him or her in defending any proceedings, civil or criminal, which relate to anything done or omitted or alleged to have been done or omitted by him or her as a member of the Committee or officer of the Society and in which judgement is given in his or her favour (or the proceedings are otherwise disposed of without any finding or admission of any material breach of duty on his or her part), or in which he or she is acquitted, or in connection with any application under any statute for relief from liability in respect of any such act or omission in which relief is granted to him or her by the Court.
\item So far as may be permitted by law, the Society may purchase and maintain for any member of the Committee or officer of the Society insurance cover against any liability which by virtue of any rule of law may attach to him or her in respect of any negligence, default, breach of duty or breach of trust of which he or she may be guilty in relation to the Society and against all costs, charges, losses and expenses and liabilities incurred by him or her and for which he or she is entitled to be indemnified by the Society by virtue of the above.
\end{enumerate}
\section{Dissolution}
\begin{enumerate}
\item The Society may be dissolved at any time by the approving votes of two-thirds of those present in person or by proxy at a General Meeting. The Society may also be dissolved (without the need for any resolution of the members) by means of not less than thirty days notice from the Proctors to the Secretary of the Society if at any time the Society ceases to be registered with the Proctors.
\item In the event of the Society being dissolved, its assets shall not be distributed amongst the members, but shall be paid to or at the direction of the The Newman Trust (registered charity number 251158).
\end{enumerate}
\section{Interpretation}
Any question about the interpretation of this Constitution shall be settled by the Committee, to be determined by vote, if necessary. If the Committee itself is divided on the interpretation of any given matter, they shall seek the input of the Chaplains. If any interpretation is still then debated, the opinion of the Senior Member is to be sought. Should the Senior Member's input be unable to resolve the matter of interpretation, the matter is to be settled by the Proctors.
\section{Schedule I -- Code of Conduct on Safety Matters}
The only risks of personal injury identified in the Society's activities are those present in the regular use of the Chaplaincy's kitchen and bar facilities. The Society will maintain its record of safety in the use of such facilities by continuing to follow the prescribed rules and advice the Chaplaincy have posted in the said facilities, observing all guidance on the safe handling, preparation, and serving of food and drink. The Society will defer to the Chaplaincy's guidance in matters of the use of the kitchen and bar pertaining to health and safety.
\newpage
\section{By-Laws of the Society}
\subsection{Membership of the Society}
\begin{enumerate}
\item All students requesting entrance into the Society by power of \nameref{con:mem} (\cref{con:mem} \cref{mem:bap}) shall be recognized as Full Members (or, simply, ``Members''). All persons admitted into the Society by means of \nameref{con:mem} (\cref{con:mem} \crefrange{mem:vis}{mem:oth}) shall be eligible for membership as Associate Members, only. However, any person shall not be admitted if the Chaplains or Committee should object to their admission within a reasonable time of the expression of the desire to be admitted to membership. All such persons as are admitted will be entitled to all rights and privileges accorded to the status of their membership in this Constitution. A person may resign membership in the Society by tendering his or her resignation in writing to the Committee, and such resignation is effective upon receipt.  The Committee may remove a person from membership - for good reason - with the approval of the Chaplains. The expelled member shall have recourse to the Senior Member if desiring to appeal against such a removal of membership.
\item Attendance at events organised by the Society and participation in the activities of any of the Chaplaincy groups affiliated to the Society is not confined to members of the Society. The Society welcomes all interested persons to join in its activities. This, however, does not apply to formal meetings or functions of the Society as described in \nameref{con:str} (\cref{con:str}), \nameref{byl:str} (\cref{byl:str}), and \nameref{byl:mee} (\cref{byl:mee}).
\item All student groups officially attached to the Chaplaincy are automatically affiliated to the Society, and are entitled to all rights and privileges accorded to such groups in this Constitution. Individual members of such groups, however, may or may not have Full or Associate membership of the Society, and groups should respect the rules of the Society in this regard.
\end{enumerate}
\subsection{Officers of the Society}\label{byl:off}
\begin{enumerate}
\item The requirement of Full Membership does not, however preclude any Associate Member from standing for election to these three particular offices. Should any Associate Member run for any of the offices of President, Secretary, or Treasurer, and be elected, their assumption of office will depend upon the Proctors' approval. Election of Associate Members to any of the other offices does not need such approval.
\item The Officers of the Society shall be the: the President; the Vice-President; the Secretary; the Treasurer; the Social Secretary; the Ecumenical Officer;the Publicity Officer; the Charities Officer; the St Thomas More Lectures Secretary; the Senior Member; the Group Representatives; the College Representatives; the Bar Manager(s) and other officers appointed from time to time by the Committee under the terms of \nameref{byl:off} (\cref{byl:off}).
\item The President, Vice-President, Secretary, Treasurer, Social Secretary, Ecumenical Officer, Publicity Officer, the St Thomas More Lectures Secretary, Charities Officer, and Senior Member, shall be termed the Executive Officers of the Society and shall constitute the Committee. The roles of the members of the Committee are as follows:
\begin{enumerate}
\item The President shall act as a representative of the Society at the direction of the Society or of the Committee; invite and act as a liaison with guest speakers for Society events; conduct urgent business of the Society in consultation with as many members of the Committee as are available if there is insufficient time to convene a meeting of the Committee or a General Meeting, but all such action shall be subject to ratification at the next meeting of the Committee or the next General Meeting; act as a liaison between the Society and the Chaplains; act as a liaison with other Catholic organisations in Oxford; oversee the operations of the Society; and carry out any other duties assigned by the Society or the Committee.
\item The Vice-President shall assist the President in execution of the President's duties; act in lieu of the President in case of the President's absence or inability to act; undertake special projects at the direction of the Committee or a General Meeting; co-ordinate College activities in consultation with the College Representatives; and carry out any other duties assigned by the Society or the Committee.
\item The Secretary shall prepare and circulate the agenda for Committee or General meetings in consultation with the members of the Committee; maintain permanent records, membership lists, and minutes of the Society; undertake the administrative and clerical duties of the Society; and carry out any other duties assigned by the Society or the Committee.
\item The Treasurer shall maintain the accounts and monies of the Society; be responsible for the allocation of funds and payment of all bills, debts, and reimbursements; collect all accounts, fees, and monies of the Society; and carry out any other duties assigned by the Society or the Committee.
\item The Social Secretary shall organise community activities of the Society; and carry out any other duties assigned by the Society or the Committee.
\item The Ecumenical Officer shall act as a liaison between the Society and other Christian organisations and religious organisations in the University; organise community activities between the Society and these organisations; and carry out any other duties assigned by the Society or the Committee.
\item The Publicity Officer shall act as a liaison between the Society and the media; maintain the social media and webpages associated with the Society and its events; prepare any necessary advertisements for the Society and its events; and carry out any other duties assigned by the Society or the Committee.
\item The Charities Officer shall act as a liaison between the Society and Catholic charities in Oxford; and carry out any other duties assigned by the Society or the Committee.
\item The St Thomas More Lectures Secretary shall be responsible for the duties outlined in Section 8; and carry out any other duties assigned by the Society or the Committee.
\end{enumerate}
\item \label{off:exe} The Executive Officers shall be elected according to the provisions made in \nameref{byl:ele} (\cref{byl:ele}) of this Constitution. The members of the Committee shall be eligible for re-election, but they may not hold office for any continuous period of more than one year without offering themselves for election.  All members may run for election to any of these offices. However, in order to hold the offices of President, Secretary, or Treasurer, the elected must be Full Members of the Society.
\item The President, Treasurer and Secretary may not hold office beyond the Michaelmas term of the final year of their course of study. Should a member of the Society apply for admission to a further course of study, which would extend their time at the University beyond the terminal year of their current course, such a person may stand for election and accept office, but, should their application be denied, they must resign their office, immediately.
\item The Group Representatives shall be the persons currently leading the various chaplaincy groups affiliated to the Society, or persons delegated by them from within these groups. They shall serve for a period of time determined by the particular group. In order to hold office they must be either Full or Associate Members of the Society.
\item The College Representatives shall act as liaisons between the Society, Chaplains, their respective College, and other College Representatives, in order to help fulfil the aims of the Society (cf., \nameref{sec:nam} (\cref{sec:nam} \cref{nam:aim})) for the students of their respective College. They shall be those persons currently serving as Chaplaincy Representatives in their respective Colleges or Associated Institutions, or persons delegated by them. They shall serve for as long as they hold the position of Chaplaincy Representative. In order to hold office they must be either Full or Associate Members of the Society.
\item The Committee may appoint Bar Manager(s) as well as a number of additional officers to portfolios which the Committee shall determine. Such appointed officers must be Full or Associate Members of the Society.
\item Any Roman Catholic member of the University Congregation is eligible to serve as Senior Member of the Society. The chosen Senior Member should be closely affiliated with the operations of the Chaplaincy. The choice of the Senior Member is to be left to the Committee, by standard vote, according to the forms of the committee meeting (\cref{byl:mee} \crefrange{mee:att}{mee:quo}) . Should the Chaplains be predominantly members of one religious institute, it is suggested that the local superior be invited to serve as Senior Member, provided that this person be a member of Congregation. The Senior Member shall serve for a term of one academic year, to begin each Michaelmas term, without any limit to the number of terms they may hold the office.
\item If some Executive Officer of the Committee relinquishes his or her post during their term of office, then a replacement may be co-opted by the Committee immediately. A by-election may be called to coincide with the next election called according to \nameref{byl:ele} (\cref{byl:ele} \crefrange{ele:dat}{ele:unu}) of this constitution and these by-laws (unless that clause dictates that an autumnal election to that post shall be held in any case, which takes priority).

Should the Committee co-opt an Associate Member to hold any of the offices of President, Secretary, or Treasurer, such an appointment will require the approval of the Proctors. (Cf., \nameref{byl:off} (\cref{byl:off}), \cref{off:exe})
\end{enumerate}
\section{Election of Officers}\label{byl:ele}
\begin{enumerate}
\item The election procedure provides for the election to the Executive Offices of the Society. Only Full and Associate Members of the Society are eligible to participate in any way in the election process.
\item \label{ele:dat} Elections to the following offices of the Committee, i.e. those of President, Vice-President, Secretary, Treasurer, Social Secretary, Ecumenical Officer, Publicity Officer, St Thomas More Lectures Secretary, and Charities Officer, shall ordinarily take place annually during the seventh week of Michaelmas term during a General Meeting.
\item \label{ele:unu} In unusual circumstances, the Committee may vary the times of the Elections specified in \cref{ele:dat}, provided the criteria for eligibility in \nameref{byl:off} (\cref{byl:off} \cref{off:exe}) is satisfied.
\item The Committee shall appoint a Returning Officer (ordinarily one of its number) for each set of elections. The Returning Officer shall be responsible for the organisation and fair conduct of the elections.
\item Notification of elections shall be published in the Chaplaincy's weekly bulletin for the fifth week of the term in which elections are to take place.
\item Nominations, with proposer and seconder, shall be submitted to the Returning Officer or the President no later than 11 PM on the Wednesday of Sixth Week, that is, the week preceding the week in which the elections are to be held. Each nomination shall be accompanied by a manifesto of not more than one hundred and fifty words (save that manifestos for the post of President or Vice-President may be longer, but not exceeding two hundred and fifty words), and a photograph of the nominee. These shall be posted up in a suitable place at the Chaplaincy from 6 PM on Friday of Sixth Week until the time of the elections.
\item The elections shall take place after hustings on the day appointed. The names of the successful candidates shall be posted in the same place in the Chaplaincy as the manifestos were displayed for one week following the day of the election.
\item The election for each post shall be conducted by the method of Single Transferable Vote. A candidate may stand for up to two posts (or three including President), and shall indicate to the Returning Officer which is his or her preferred position. If elected to both positions stood for, a candidate's votes will be transferred to the remaining candidates for the second-choice post. Should it not be possible to separate two or more candidates, then the President shall have a casting vote.
\item If only one candidate submits him/herself for election to a given post then that candidate shall be returned unopposed without need for a vote.If some Executive Officer whose term of office is not due to conclude stands for election to some other office of the committee at a given election, then a by-election (which he/she may contest) must be held for the post which he/she currently holds.
\item The officers of the Committee shall ordinarily hold office until the end of Michaelmas term. Following their election, the officers-elect of the Committee shall attend meetings of the Committee during the rest of Michaelmas term. The term of office of the incoming Committee shall ordinarily commence at the beginning of Hilary term.
\item No member of the existing Committee shall propose or second any candidate.
\item No candidate shall propose or second another candidate.
\item The Returning Officer shall place a mark next to the name of each member of the Society on the Society's membership list at the time that this member is given a ballot paper.
\end{enumerate}
\section{Committe Structure}\label{byl:str}
There shall be a sub-committee for the St Thomas More Lectures (cf., \nameref{his:fou} (\cref{sec:nam} \cref{his:fou})) which shall not be concerned with the government of the Society but shall only manage the St Thomas More Lectures and the Mass and any Dinner associated with each St Thomas More Lecture.

The Sub-Committee for the St Thomas More Lectures shall consist of the President, the Treasurer, the St Thomas More Lectures Secretary, the Senior Chaplain, and two other members of the University appointed by the Senior Chaplain.
\section{Meetings of the Committee}\label{byl:mee}
\begin{enumerate}
\item The Committee shall ordinarily meet three times during each University Full Term. The Senior Member is not required to attend any of these meetings.
\item \label{mee:att} These meetings shall be attended only by the Committee. Occasionally the President may invite the Chaplains or other persons to attend, without the right to vote, if the Committee should agree that this may assist the business of the meeting.
\item These meetings shall have a proper Agenda, and the Minutes shall be recorded and distributed to each member of the Committee and the Chaplains. The drafting and dissemination of these documents are among the duties of the Secretary, or someone else assigned by the chair of the meeting.
\item These meetings shall be chaired by the President, if he or she is present. In the President's absence, the meeting will be chaired by the Vice President. If both President and Vice-President be absent, another member of the Committee shall be delegated to the chair, to be selected by those present.
\item Meetings shall be conducted in accordance with the norms of Committee practice.
\item \label{mee:quo} For the purposes of making a decision, the Meeting must have a Quorum. This requires seven members of the Committee to be present, of whom one must either the President or Vice-President. For a resolution to be passed, a majority of those present must vote in favour.
\item To deal with matters of urgency, the President may call an Extra-Ordinary Meeting at any other time. Such a meeting must abide by the provisions laid down in \crefrange{mee:att}{mee:quo}. Notice and documentation of any such Extra-Ordinary Meetings will be the responsibility of the Secretary.
\item In exceptional circumstances the Senior Member may veto a decision of the Committee. 
\end{enumerate}
\section{General Meetings of the Society}
\begin{enumerate}
\item A General Meeting may act in place of the Committee in any of the Committee's duties. Where a General Meeting's resolution conflicts with that of the Committee, the Committee will implement the General Meeting's resolution.
\item A General Meeting will be competent to amend the constitution only when it is called with twenty-one days' notice to members.
\item No proxies are permitted for any General Meeting of the Society. Only those present may cast votes.
\item The President of the Society shall, if present, preside over the General Meetings of the Society. If the President is not present or unable to act, the Vice-President shall preside. If neither the President nor Vice-President are present, the General Meeting shall be rearranged for an alternative date, unless there is no President or Vice-President within the committee at which point, the meeting shall select a chair \emph{pro tem} to preside. 
\item The Society shall adopt such procedures as it sees fit to operate at a General Meeting, provided that business is determined by majority voting, unless specified otherwise in this constitution or these by-laws or in motions adopted by the General Meeting.
\item The Senior Member may, on the recommendation of the Committee, or a member of the society, or of the Chaplains veto any decision of a General Meeting which he believes to threaten the Society's successful fulfilment of its aims.
\end{enumerate}
\section{The Thomas More Lectures}
\begin{enumerate}
\item The Sub-Committee for the St Thomas More Lectures (defined above)(in this clause ``the Lectures Committee'') shall organise on behalf of the Society lectures by persons of international eminence on subjects of Catholic interest to take place at least once and not more than twice in each academic year. The lectures shall be known as St Thomas More Lectures.
\item The St Thomas More Lectures Secretary shall be convener, secretary, and Executive Officer of the Lectures Committee and the Senior Chaplain shall be its chairman. The Lectures Committee shall meet as often as necessary to transact its business but in any case not less often than once a term. Between meetings it shall be permissible for decisions to be taken by majority vote by email.
\item Each St Thomas More Lecture shall be preceded or followed (as may be most convenient) by the celebration of Holy Mass. 
\item The Lectures Committee may also organise a dinner in honour of each St Thomas More Lecturer.
\item The Lectures Committee may, in co-ordination with the Chaplains, the Committee, and the trustees of the Newman Trust seek financial support or sponsorship for the St Thomas More Lectures or for any individual St Thomas More Lecture.
\end{enumerate}
\section{The Constitution}
\begin{enumerate}
\item This Constitution may be amended from time to time in the light of experience. Such amendments are made by the procedure outlined in \crefrange{con:cha}{con:app}.
\item \label{con:cha} A copy of any proposed amendments shall be sent to all Executive Officers. The Committee shall carefully examine the proposals and take a formal vote on them. Voting shall be in accordance with \nameref{byl:mee} (\cref{byl:mee}, \cref{mee:quo}).
\item If approved by the Committee, the proposed amendment shall be passed to the Chaplains.
\item If approved by the Chaplains, the proposed amendment will be made available for inspection by members of the Society. At a meeting of the general Society, a formal vote shall be taken by all Full and Associate members of the Society there present, only, on the proposed amendment.
\item \label{con:app} If approved (by simple majority), then the Constitution shall be deemed to have been amended; a copy of the amended Constitution shall be deposited with the Proctors.
\end{enumerate}

\centerline{\scshape{LDS -- Laus Deo Semper}}

\centerline{\scshape{Praise be God Forever}}
\end{document}
